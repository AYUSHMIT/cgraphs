Message passing is a useful abstraction for implementing concurrent
programs.
For real-world systems, however, it is often combined with other programming
and concurrency paradigms, such as
higher-order functions, mutable state, shared-memory concurrency, and locks.
We present \textbf{\lname}: a logic for proving functional correctness of
programs that use a combination of the aforementioned features.
\lname combines the power of modern concurrent separation logics with a
first-class protocol mechanism---based on session types---for
reasoning about message passing in the presence of other concurrency paradigms.
We show that \lname provides a suitable level of abstraction by proving
functional correctness of a variety of examples, including
a distributed merge sort, a distributed load-balancing mapper, and a variant
of the map-reduce model, using concise specifications.

While \lname was already presented in a conference paper (POPL'20), this paper
expands the prior presentation significantly.
Moreover, it extends \lname to \textbf{\lname 2.0} with a notion of
\emph{subprotocols}---based on session-type subtyping---that permits additional flexibility when
composing channel endpoints, and that takes full advantage of the asynchronous
semantics of message passing in Actris.
Soundness of \lname 2.0 is proved using a model of its protocol mechanism in the
Iris framework.
We have mechanised the theory of \lname, together with custom tactics, as well as
all examples in the paper, in the Coq proof assistant.
